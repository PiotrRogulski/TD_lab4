\documentclass[a4paper,12pt,notitlepage]{article}

\usepackage{amsmath, amssymb, amsthm}
\usepackage{mathtools}
\usepackage[math]{fontspec}
\usepackage[nosingleletter, lastparline]{impnattypo}
\usepackage[polish]{babel}
\usepackage{bookmark}
\usepackage[margin=1in]{geometry}
\usepackage[babel=true, tracking=true]{microtype}
\usepackage{minted}
\usepackage{parskip}
\usepackage{array}
\usepackage{colortbl}
\usepackage{hhline}

\linespread{1.3}

\renewcommand*{\thesection}{\Alph{section}}

\definecolor{bg}{rgb}{0.95,0.95,0.95}
\setminted{frame=single, bgcolor=bg, breaklines=true, autogobble}

\IfFontExistsTF{JetBrainsMono-Regular}{
    \setmonofont{JetBrainsMono}[
        UprightFont = *-Light,
        BoldFont = *-Regular,
        ItalicFont = *-Light-Italic,
        Scale = MatchLowercase
    ]
}{}

\title{\textbf{TD -- laboratorium 4}}
\author{Piotr Rogulski 305867 \\ Szymon Sieradzki 305881}
\date{\today}

\begin{document}

\maketitle

\section{Adresacja}

\begin{table}[htbp]
    \centering
    \caption{Adresacja interfejsów}
    \begin{tabular}{*4c}
        \hline\hline
            \textbf{Router} & \textbf{Interfejs} & \textbf{Adres IP} & \textbf{Podsieć} \\
        \hline
            R1 & e0/0  & 10.0.12.1 & 10.0.12.0/30 \\
        \hline
            R2 & e0/0  & 10.0.12.2 & 10.0.12.0/30 \\
               & e0/1  & 10.0.24.1 & 10.0.24.0/30 \\
               & e0/2  & 10.0.23.1 & 10.0.23.0/30 \\
        \hline
            R3 & e0/0  & 10.0.35.1 & 10.0.35.0/30 \\
               & e0/2  & 10.0.23.2 & 10.0.23.0/30 \\
               & e0/3  & 10.0.34.1 & 10.0.34.0/30 \\
        \hline
            R4 & e0/1  & 10.0.24.2 & 10.0.24.0/30 \\
               & e0/2  & 10.0.45.1 & 10.0.45.0/30 \\
               & e0/3  & 10.0.34.2 & 10.0.34.0/30 \\
        \hline
            R5 & e0/0  & 10.0.35.2 & 10.0.35.0/30 \\
               & e0/2  & 10.0.45.2 & 10.0.45.0/30 \\
        \hline\hline
    \end{tabular}
\end{table}

\begin{table}[htbp]
    \caption{Adresacja interfejsów loopback}
    \centering
    \begin{tabular}{*2c}
        \hline\hline
            \textbf{Router} & \textbf{Loopback IP} \\
        \hline
            R1 & 1.1.1.1 \\
            R2 & 2.2.2.2 \\
            R3 & 3.3.3.3 \\
            R4 & 4.4.4.4 \\
            R5 & 5.5.5.5 \\
        \hline\hline
    \end{tabular}
\end{table}

\section{Protokół OSPF}

Zanim skonfigurowany zostanie MPLS, w sieci musi być uruchomiony protokół trasowania (OSPF). W tym celu, na każdym routerze należy uruchomić proces OSPF i dodać do niego odpowiednie podsieci (w tym podsieci loopback):
\begin{minted}[label=Konfiguracja OSPF]{text}
    #router ospf 1
    #network <adres ip sieci> <maska wildcard> area 0
\end{minted}

Wszystkie interfejsy (oprócz interfejsów loopback) powinny pracować w trybie point-to-point. Jest to realizowane za pomocą polecenia \mintinline{text}{ip ospf network point-to-point}. Koszt łącza między routerami R2 oraz R3 został ustawiony na wartość 100 a wszystkie pozostałe na wartość 10 za pomocą polecenia \mintinline{text}{ip ospf cost <koszt>}.

% TODO: output traceroute z R1 do R5 + wyjaśnienie

\section{Protokół LDP}

W celu zwiększenia wydajności, przed uruchomieniem protokołu MPLS włączony został Cisco Express Forwarding. MPLS należy uruchomić i w kontekście globalnym i w kontekście każdego z interfejsów.
\begin{minted}[label=Konfiguracja MPLS]{text}
    #ip cef
    #mpls ip
    #interface <numer interfejsu>
    #mpls ip
\end{minted}

\section{Weryfikacja MPLS/LDP}

Po ustawieniu MPLS należy upewnić się, że konfiguracja przebiegła pomyślnie używając następujących poleceń:
\begin{minted}[label=Weryfikacja konfiguracji MPLS i LDP]{text}
    #show mpls interfaces
    #show mpls ldp neighbor
    #show mpls ldp binding
    #show mpls forwarding-table
    #show mpls forwarding-table <prefix> detail
\end{minted}

% TODO: output poleceń na R2 + wyjaśnienie

\section{Weryfikacja ścieżek}

\section{Podstawowe funkcje inżynierii ruchu}



\end{document}
