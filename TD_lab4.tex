\documentclass[a4paper,12pt,notitlepage]{article}

\usepackage{amsmath, amssymb, amsthm}
\usepackage{mathtools}
\usepackage[math]{fontspec}
\usepackage[nosingleletter, lastparline]{impnattypo}
\usepackage[polish]{babel}
\usepackage{bookmark}
\usepackage[margin=1in]{geometry}
\usepackage[babel=true, tracking=true]{microtype}
\usepackage{minted}
\usepackage{parskip}
\usepackage{array}
\usepackage{colortbl}
\usepackage{hhline}

\linespread{1.3}

\renewcommand*{\thesection}{\Alph{section}}

\definecolor{bg}{rgb}{0.95,0.95,0.95}
\setminted{frame=single, bgcolor=bg, breaklines=true, autogobble}

\IfFontExistsTF{JetBrainsMono-Regular}{
    \setmonofont{JetBrainsMono}[
        UprightFont = *-Light,
        BoldFont = *-Regular,
        ItalicFont = *-Light-Italic,
        Scale = MatchLowercase
    ]
}{}

\title{\textbf{TD -- laboratorium 4}}
\author{Piotr Rogulski 305867 \\ Szymon Sieradzki 305881}
\date{\today}

\begin{document}

\maketitle

\section{Adresacja}

\begin{table}[htbp]
    \centering
    \caption{Adresacja interfejsów}
    \begin{tabular}{*4c}
        \hline\hline
            \textbf{Router} & \textbf{Interfejs} & \textbf{Adres IP} & \textbf{Podsieć} \\
        \hline
            R1 & e0/0  & 10.0.12.1 & 10.0.12.0/30 \\
        \hline
            R2 & e0/0  & 10.0.12.2 & 10.0.12.0/30 \\
               & e0/1  & 10.0.24.1 & 10.0.24.0/30 \\
               & e0/2  & 10.0.23.1 & 10.0.23.0/30 \\
        \hline
            R3 & e0/0  & 10.0.35.1 & 10.0.35.0/30 \\
               & e0/2  & 10.0.23.2 & 10.0.23.0/30 \\
               & e0/3  & 10.0.34.1 & 10.0.34.0/30 \\
        \hline
            R4 & e0/1  & 10.0.24.2 & 10.0.24.0/30 \\
               & e0/2  & 10.0.45.1 & 10.0.45.0/30 \\
               & e0/3  & 10.0.34.2 & 10.0.34.0/30 \\
        \hline
            R5 & e0/0  & 10.0.35.2 & 10.0.35.0/30 \\
               & e0/2  & 10.0.45.2 & 10.0.45.0/30 \\
        \hline\hline
    \end{tabular}
\end{table}

\begin{table}[htbp]
    \caption{Adresacja interfejsów loopback}
    \centering
    \begin{tabular}{*2c}
        \hline\hline
            \textbf{Router} & \textbf{Loopback IP} \\
        \hline
            R1 & 1.1.1.1 \\
            R2 & 2.2.2.2 \\
            R3 & 3.3.3.3 \\
            R4 & 4.4.4.4 \\
            R5 & 5.5.5.5 \\
        \hline\hline
    \end{tabular}
\end{table}

\section{Protokół OSPF}

Zanim skonfigurowany zostanie MPLS, w sieci musi być uruchomiony protokół trasowania (OSPF). W tym celu, na każdym routerze należy uruchomić proces OSPF i dodać do niego odpowiednie podsieci (w tym podsieci loopback):
\begin{minted}[label=Konfiguracja OSPF]{text}
    #router ospf 1
    #network <adres ip sieci> <maska wildcard> area 0
\end{minted}

Wszystkie interfejsy (oprócz interfejsów loopback) powinny pracować w trybie point-to-point. Jest to realizowane za pomocą polecenia \mintinline{text}{ip ospf network point-to-point}. Koszt łącza między routerami R2 oraz R3 został ustawiony na wartość 100 a wszystkie pozostałe na wartość 10 za pomocą polecenia \mintinline{text}{ip ospf cost <koszt>}.

Korzystając z  \mintinline{text}{traceroute 5.5.5.5} można ustalić trasę OSPF z R1 do R5. Ponieważ koszt łącza R2-R3 jest ustawiony na 100 to trasa przebiega przez R1-R2-R4-R5.
\inputminted[label=Traceroute R1 do R5, firstline=175, lastline=182]{text}{R/R1.txt}

\section{Protokół LDP}

W celu zwiększenia wydajności, przed uruchomieniem protokołu MPLS włączony został Cisco Express Forwarding. MPLS należy uruchomić zarówno i w kontekście globalnym, jak i w kontekście każdego z interfejsów.
\begin{minted}[label=Konfiguracja MPLS]{text}
    #ip cef
    #mpls ip
    #interface <numer interfejsu>
    #mpls ip
\end{minted}

\section{Weryfikacja MPLS/LDP}

Po ustawieniu MPLS należy upewnić się, że konfiguracja przebiegła pomyślnie używając następujących poleceń:
\begin{minted}[label=Weryfikacja konfiguracji MPLS i LDP]{text}
    #show mpls interfaces
    #show mpls ldp neighbor
    #show mpls ldp binding
    #show mpls forwarding-table
    #show mpls forwarding-table <prefix> detail
\end{minted}

Poniżej przedstawione są wywołania tych komend na routerze R2. Wywołanie polecenia \mintinline{text}{show mpls interfaces} pokazuje interfejsy R2 objęte przez MPLS, czyli zgodnie z ustawieniami wszystkie interfejsy R2. Komenda \mintinline{text}{show mpls ldp neighbor} pokazuje informacje o sąsiadach LDP routera R2, są to routery R1, R3 i R4 o LDP identyfikatorach odpowiednio 1.1.1.1, 3.3.3.3 i 4.4.4.4. Wpisy w bazie LIB przedstawione są przez wywołanie \mintinline{text}{show mpls ldp neighbor}, są to wszystkie podsieci w sieci. Bazę LFIB pokazuje z kolei wywołanie \mintinline{text}{show mpls forwarding-table}.
\inputminted[label=Weryfikacja MPLS/LDP na R2, firstline=204, lastline=307]{text}{R/R2.txt}

\section{Weryfikacja ścieżek}

Analizując forwarding table na R1, w drodze do R5 następnym skokiem jest 10.0.12.2, z forwarding table na R2 wiadomo, że następnym skokiem do R5 jest 10.0.24.2, z R4 następnym skokiem jest 10.0.45.2. Zgadza się to z trasą pokazywaną przez traceroute. Z analizy forwarding table na routerach R5 jest oznaczony (na każdym routerze) przez etykietę 19.
\inputminted[label=Forwarding table R1, firstline=253, lastline=272]{text}{R/R1.txt}

\section{Podstawowe funkcje inżynierii ruchu}

By wdrożyć MPLS TE należy skonfigurować router id na adres loopback i włączyć MPLS TE dla protokołu routingu przy pomocy poleceń \mintinline{text}{mpls traffic-eng area 0} oraz \mintinline{text}{mpls traffic-eng <router id> loopback0}. W dalszej kolejności funkcję inżynierii ruchu trzeba również włączyć na interfejsach i w konfiguracji globalnej, przy pomocy \mintinline{text}{mpls-traffic-eng tunnels}

W celu uruchomienia protokołu RSVP należy na każdym interfejsie wywołać polecenie \mintinline{text}{ip rsvp bandwidth <wartość>}, gdzie <wartość> to przepustowość łącza. Na wszystkich interfejsach poza tym łączącym R2 i R4 przepustowość została ustawiona na 512 kb/s, dla połączenia R2-R4 została ustawiona na 64 kb/s.

Między routerami R1 i R5 został stworzony dynamiczny tunel RSVP TE o przepustowości 256. Na routerze R1 docelowy adres tunelu został ustawiony na adres loopback R5 (5.5.5.5), na R5 adres ten został ustawiony na adres loopback R1, tj. 1.1.1.1.

Stworzony tunel R1-R5 wiedzie przez 10.0.12.2, 10.0.23.2 i 10.0.35.2, tj. routery R2, R3 i R5.
\inputminted[label=MPLS tunel R1-R5, firstline=309, lastline=357]{text}{R/R1.txt}

Z analizy tablicy routingu OSPF i tablicy komutacji etykiet MPLS wynika, że dane z R1 do R5 istotnie są przesyłane przez stworzony tunel (oznaczony w interfejsach jako Tunnel1). Trasa tego tunelu nie wiedzie tak jak w poprzednim przypadku przez R1-R2-R4-R5, ponieważ przepustowość na łączu R2-R4 jest ograniczona do 64 kb/s.
\inputminted[label=show ip route i show mpls forwarding-table dla R1, firstline=362, lastline=406]{text}{R/R1.txt}

Wywołanie \mintinline{text}{traceroute 5.5.5.5} z R1 potwierdza przesyłanie danych nowo utworzonym tunelem.
\inputminted[label=traceroute 5.5.5.5 z R1 po utworzeniu tunelu, firstline=407, lastline=414]{text}{R/R1.txt}

Po zmianie przepustowości R2-R4 na 512 kb/s i zmianie przepustowości R2-R3 na 64 kb/s, tunel R1-R5 zmienia swoją trasę na R1-R2-R4-R5. Potwierdza to wywołanie \mintinline{text}{traceroute 5.5.5.5} na R1 po zmianie.
\inputminted[label=traceroute 5.5.5.5 z R1 po zmianie przepustowości, firstline=425, lastline=432]{text}{R/R1.txt}

W dalszej części zamknięty zostaje tunel R1-R5, wówczas trasa R1-R5 wyznaczana jest przez OSPF, tj. na podstawie najmniejszej metryki, w tym przypadku pozostaje ona bez zmian, tj. R1-R2-R4-R5.
\inputminted[label=Zamknięcie tunelu R1-R5, firstline=439, lastline=472]{text}{R/R1.txt}

Stworzony zostaje nowy tunel R1-R5 z explicit-path, tj. tunel, którego trasa jest z góry narzucona. Konfiguracji kolejnych adresów trasy tunelu można dokonać poprzez polecenie \mintinline{text}{ip explicit-path name <nazwa> enable}. Trasa zostaje ustalona na R1-R2-R3-R4-R5, z wywołania \mintinline{text}{traceroute 5.5.5.5} dane przesyłane są powstałym tunelem.
\inputminted[label=Tunel R1-R5 z explicit-path R1-R2-R3-R4, firstline=521, lastline=538]{text}{R/R1.txt}

Komenda \mintinline{text}{show ip ospf mpls traffic-eng link} pozwala ustalić, do jakich podsieci dany router rozgłasza, w przypadku wywołania na R2 otrzymujemy informacje o rozgłaszaniu przez R2 do podsieci 10.0.12.0/30, 10.0.24.0/30 i 10.0.23.0/30.
\inputminted[label=show ip ospf mpls traffic-eng link na R2, firstline=456, lastline=508]{text}{R/R2.txt}

Architektura MPLS pozwala również na stworzenie tunelu z luźnymi segmentami. Tunel taki został utworzy z routera R1 do R5 z luźnym segmentem od R5 do R2. Niestety wersja oprogramowania na routerach nie pozwala na uaktywnienie tego tunelu. Po stworzeniu tunelu z loose hops jest on zdefiniowany (\mintinline{text}{Admin: up}), lecz nieaktywny (\mintinline{text}{Oper: down}).
\inputminted[label=Tunel R5-R1 z loose hops, firstline=1266, lastline=1295]{text}{R/R5.txt}
\end{document}
